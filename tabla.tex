% Ejemplo de documento LaTeX
% Tipo de documento y tamaño de letra
\documentclass[12pt]{article}
% Preparando para documento en Español.
% Para documento en Inglés no hay que hacer esto.
\usepackage[spanish]{babel}
\selectlanguage{spanish}
\usepackage[utf8]{inputenc}
% EL titulo, autor y fecha del documento
\title{Comandos Bash}
\author{Linus Torsvald}
\date{03 de Febrero de 2015}
% Aqui comienza el cuerpo del documento
\begin{document}
% Construye el título
\maketitle
\section{Comandos {\tt bash}}
Bash es un interpretador de comandos utilizado sobre el sistema operativo Linux.
Su función es de mediar entre el usuario y el sistema.
\section{Navegación}
Esta sección describe como se puede navegar entre archivos y directorios.
\section{Husmeando en el sistema}
Veremos 3 comandos:
\begin{itemize}
\item {\tt ls} (Lista los archivos y directorios)
\item {\tt less} (Ver el contenido de archivos)
\item {\tt file} (Nos informa sobre el tipo de archivo)
\end{itemize}
\begin{enumerate}
\item {\tt ls} (Lista los archivos y directorios)
\item {\tt less} (Ver el contenido de archivos)
\item {\tt file} (Nos informa sobre el tipo de archivo)
\end{enumerate}
\definecolor{micolor}{rgb}{0,1,0.5}
\begin{tabular}{
>{\columncolor[cmyk]{0.8,0.5,0.4,0.1}}c |
>{\columncolor[gray]{0.7}}c |
>{\columncolor{micolor}} c|}
\hline
 & Comando & Funcion & Ejemplo \\
\hline
& c & gray & definido por nosotros\\
\hline
\end{tabular} 
\section{Y otra sección mas}
\subsection{Y una subsección}
% Nunca debe faltar esta última linea.
\end{document}
