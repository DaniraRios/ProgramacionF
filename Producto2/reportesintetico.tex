% Ejemplo de documento LaTeX
% Tipo de documento y tamaño de letra
\documentclass[10pt]{article}
% Preparando para documento en Español.
% Para documento en Inglés no hay que hacer esto.
\usepackage[spanish]{babel}
\selectlanguage{spanish}
\usepackage[utf8]{inputenc}
% EL titulo, autor y fecha del documento
\title{Compiladores e Interpretadores}
\author{Rios Quijada Danira}
\date{15 de Febrero de 2015}
% Aqui comienza el cuerpo del documento
\begin{document}
% Construye el título
\maketitle
\section{Introducción}
Un compilador analiza el programa y lo traduce al idioma "maquina". La acción fundamental los compiladores es equivalente a la de un traductor humano , que toma nota de lo que esta escuchando y reproduce por escrito en otra lengua.Por otro lados los interpretadores analizan el programa fuente y lo ejecutan directamente, o sea en el ejemplo del traductor humano, éste sería un traductor humano que conforme a lo que está escuhando va ejecutando, sin generar ningun escrito, es decir que sobre la marcha va traduciendo.
 
\section{Tabla comparativa entre distintos lenguajes de programación}
\begin{tabular}{||l|l|l|l|l||}
\hline \hline
Nombre & Paradigma & Creadores & Año & Extención \\ \hline
C & Imperativo & Dennis M. Ritchie & 1972 & .c \\ \hline
C++ & Imperativo, orientado a objetos & Bjarne Stroustrup & 1983 & .cpp \\ \hline
Java & Orientado a objetos, imperativo & Sun Microsystems & 1995 & .java \\ \hline
Fortran & Imperativo & IBM & 1957 & .f90 (Depende/versión)\\ \hline
Python & Funcional, reflexivo, O.O & Guido van Rossum & 1991 & .py \\ \hline
Ruby & O.O, reflexivo & Yukihiro Matsumoto & 1995 & .rb \\ \hline \hline
\end{tabular}
\section{Ejemplo del programa Adivina el resultado de las operaciones mentales al haber escogido un número previamente y de común acuerdo con el usuario, en cada uno de los lenguajes descritos}
\subsection{C}
\begin{verbatim}
#include <stdio.h>
#include <unistd.h> 
int main()
{
    printf("Hola! Trataré de adivinar un número.\n");

 printf("Piensa un número entre 1 y 10.\n");
    sleep(5); 
    printf("Ahora multiplícalo por 9.\n");
    sleep(5); 
    printf("Si el número tiene 2 dígitos, súmalos entre si: Ej. 36 -> 3+6=9. Si tu número tiene un solo dígito, súmale 0.\n");
    sleep(5); 
    printf("Al número resultante súmale 4.\n");
    sleep(10); 
    printf("Muy bien. El resultado es 13 :) \n");


}
\end{verbatim}
\subsection{C++}
\begin{verbatim}
 #include <iostream>
	#include <unistd.h>
	 
	int main()
	{
	  std::cout <<"Hola! Trataré de adivinar un número.\n";
	  std::cout<<"Piensa en un número entre 1 y 10\n";
	  sleep(5);
	  std::cout<<"Ahora multiplícalo por 9.\n";

          sleep(5);
          std::cout << "Si el número tiene 2 dígitos, súmalos entre si: Ej. 36 -> 3+6=9. Si tu número tiene un solo dígito, súmale 0.\n";

          sleep(5);
          std::cout << "Al número resultante súmale 4.\n";
       
          sleep(10);
          std::cout << "Muy bien. El resultado es 13 :D\n";

return(0);
}
\end{verbatim}
\subsection{Java}
\begin{verbatim} 
// Hola mundo en Java
class Holamundo {
static public void main( String args[] ) {
System.out.println("Hola! Trataré de adivinar un número.");
System.out.printIn("Piensa un número entre 1 y 10.");
try {
Thread.sleep(4000);
} catch(InterruptedException ex) {
Thread.currentThread().interrupt();
}
System.out.println("Ahora multiplícalo por 9.");
try {
Thread.sleep(4000);
} catch(InterruptedException ex) {
Thread.currentThread().interrupt();
}
System.out.println("Si el número tiene 2 dígitos, súmalos entre si: Ej. 36 -> 3+6=9. Si tu número tiene un solo dígito, súmale 0.");
try {
Thread.sleep(4000);
} catch(InterruptedException ex) {
Thread.currentThread().interrupt();
}
System.out.println("Al número resultante súmale 4.");
try {
Thread.sleep(8000);
} catch(InterruptedException ex) {
Thread.currentThread().interrupt();
}
System.out.println("Muy bien. El resultado es 13 [:");
}
\end{verbatim}
\subsection{Fortran}
\begin{verbatim} 
program hola
write(*,*) 'Hola! Trataré de adivinar un número.'
write(*,*) 'Piensa un número entre 1 y 10.'
call sleep(5)
write(*,*) 'Ahora multiplícalo por 9.'
call sleep(5)
write(*,*) 'Si el número tiene 2 dígitos, súmalos entre si: Ej. 36 -> 3+6=9. Si tu número tiene un solo dígito, súmale 0.'
call sleep(5)
write(*,*) 'Al número resultante súmale 4.'
call sleep(10)
write(*,*) 'Muy bien. El resultado es 13 n.n'
end program hola
\end{verbatim}
\subsection{Python}
\begin{verbatim}
# Hola mundo en Python
# -*- coding: 850 -*-
import time
print "Hola! Trataré de adivinar un número."
print "Piensa un número entre 1 y 10."
import time
time.sleep(5)
print "Ahora multiplícalo por 9."
import time
time.sleep(5)
print "Si el número tiene 2 dígitos, súmalos entre si: Ej. 36 -> 3+6=9. Si tu número tiene un solo dígito, súmale 0."
import time
time.sleep(5)
print "Al número resultante súmale 4."
import time
time.sleep(10)
print "Muy bien. El resultado es 13 :P "
\end{verbatim}

\subsection{Ruby}
\begin{verbatim}
# -*- coding: utf-8 -*-
# Hola mundo en Ruby
#encoding: utf-8
puts "Hola! Trataré de adivinar un número."
puts "Piensa un número entre 1 y 10."
sleep(5)
puts "Ahora multiplícalo por 9."
sleep(5)
puts "Si el número tiene 2 dígitos, súmalos entre si: Ej. 36 -> 3+6=9. Si tu número tiene un solo dígito, súmale 0."
sleep(5)
puts "Al número resultante súmale 4."
sleep(10)
puts "Muy bien. El resultado es 13 :V"
\end{verbatim}
% Nunca debe faltar esta última linea.
 \end{document}

