% Ejemplo de documento LaTeX
% Tipo de documento y tamaño de letra
\documentclass[10pt]{article}
% Preparando para documento en Español.
% Para documento en Inglés no hay que hacer esto.
\usepackage[spanish]{babel}
\selectlanguage{spanish}
\usepackage[utf8]{inputenc}
% EL titulo, autor y fecha del documento
\title{Tabla de comandos Bash}
\author{Rios Quijada Danira}
\date{03 de Febrero de 2015}
% Aqui comienza el cuerpo del documento
\begin{document}
% Construye el título
\maketitle
\begin{tabular}{||l|l|l||}
\hline \hline
Comando & Descripción & Ejemplo \\
\hline
ls & Enlista el contenido del directorio. & Ej. ls-a \\ \hline
man & Interfaz de los manuales de referencia electrónicos. & Ej. man ls \\ \hline
echo & Repite una línea de texto. & Ej. echo -n \\ \hline
pwd & Muestra el nombre del directorio actual. & Ej. pwd -P \\ \hline
cd & Cambiar de locación. & Ej. cd notas \\ \hline
file & Determina el tipo de archivo. & Ej. file-m \\ \hline
mkdir & Crear un nuevo directorio. & Ej. mkdir notas \\ \hline
rmdir & Eliminar directorios completos. & Ej. rmdir notas \\ \hline
touch & Crear archivo en blanco. & Ej. touch notas.txt \\ \hline
cp & Copia archivos y directorios. & Ej. notas notas.txt  \\ \hline
mv & Mover o renombrar archivos. & Ej. mv notas.txt aaa \\ \hline
rm & Remover directorios o archivos. & Ej. rm notas.txt \\ \hline
vi & Editor de texto. &  \\ \hline
ZZ & Salvar y salir. & \\ \hline
:w & Salvar pero no salir.& \\ \hline
cat & Vacia el contenido de un archivo. & Ej. cat notas.txt \\ \hline
less & Vacia el contenido de archivos grandes. & Ej. less notas.txt \\ \hline
ls-l & Ver los permisos de un archivo. &  Ej. ls-l notas.txt\\ \hline
chmod & Muestra los permisos de un archivo y permite cambiarlos. & Ej. chmod notas.txt \\ \hline
head & Muestra la primera parte de los archivos. & Ej. head notas.txt \\ \hline
tail & Muestra la última parte de los archivos. & Ej. tail notas.txt \\ \hline
sort & Ordena las líneas de un archivo de texto. & Ej. sort notas.txt \\ \hline
nl & Muestra el número de líneas de un archivo. & Ej. nl notas.txt \\ \hline
cut & Remueve secciones de cada línea del archivo. & \\ \hline
sed & Editor para filtrar y transformar texto. & \\ \hline
uniq & Reporta u omite líneas repetidas. & Ej. uniq notas.txt \\ \hline
tac & Vacia e imprime archivos en reversa. & Ej. tac notas.txt \\ \hline
history & Muestra los últimos comandos utilizados. & \\ \hline
reboot & Reinicia el equipo. & \\ \hline
halt & Apaga el equipo. & \\ \hline
chmod & Se utiliza para cambiar los permisos de un archivo. & \\ \hline
clear & Limpia la pantalla y coloca el prompt al principio de la misma. & \\ \hline \hline
\end{tabular}

% Nunca debe faltar esta última linea.
\end{document}
