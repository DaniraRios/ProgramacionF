% Ejemplo de documento LaTeX
% Tipo de documento y tamaño de letra
\documentclass[10pt]{article}
% Preparando para documento en Español.
% Para documento en Inglés no hay que hacer esto.
\usepackage[spanish]{babel}
\selectlanguage{spanish}
\usepackage[utf8]{inputenc}
% EL titulo, autor y fecha del documento
\title{Tabla de comandos Bash}
\author{Rios Quijada Danira}
\date{03 de Febrero de 2015}
% Aqui comienza el cuerpo del documento
\begin{document}
% Construye el título
\maketitle
\section{Comandos de ayuda}
\begin{tabular}{||l|l|l||}
\hline \hline
Comando & Descripción & Ejemplo \\
\hline
man & Interfaz de los manuales de referencia electrónicos. & Ej. man ls \\ \hline
info & Provee información del comando indicado. & Ej. info man \\ \hline
-- help & Da ayuda de los comandos. & Ej. --help ls \\ \hline
\end {tabular}
\section{Comandos para archivos y directorios}
\begin{tabular}{||l|l|l||}
\hline \hline
Comando & Descripción & Ejemplo \\ \hline
ls & Lista los archivos y  directorios. & Ej. ls -al \\ \hline
pwd & Muestra el nombre del directorio actual. & Ej. pwd -P \\ \hline
cd & Cambiar de locación. & Ej. cd notas \\ \hline
file & Determina el tipo de archivo. & Ej. file-m \\ \hline
mkdir & Crear un nuevo directorio. & Ej. mkdir notas \\ \hline
rmdir & Eliminar directorios completos. & Ej. rmdir notas \\ \hline
touch & Crear archivo en blanco. & Ej. touch notas.txt \\ \hline
cp & Copia archivos y directorios. & Ej. notas notas.txt  \\ \hline
mv & Mover o renombrar archivos. & Ej. mv notas.txt aaa \\ \hline
rm & Remover directorios o archivos. & Ej. rm notas.txt \\ \hline
cat & Vacia el contenido de un archivo. & Ej. cat notas.txt \\ \hline
less & Vacia el contenido de archivos grandes. & Ej. less notas.txt \\ \hline
ls-l & Ver los permisos de un archivo. &  Ej. ls-l notas.txt\\ \hline
chmod & Muestra los permisos de un archivo y permite cambiarlos. & Ej. chmod notas.txt \\ \hline
head & Muestra la primera parte de los archivos. & Ej. head notas.txt \\ \hline
tail & Muestra la última parte de los archivos. & Ej. tail notas.txt \\ \hline
sort & Ordena las líneas de un archivo de texto. & Ej. sort notas.txt \\ \hline
nl & Muestra el número de líneas de un archivo. & Ej. nl notas.txt \\ \hline
cut & Remueve secciones de cada línea del archivo. & \\ \hline
sed & Editor para filtrar y transformar texto. & \\ \hline
uniq & Reporta u omite líneas repetidas. & Ej. uniq notas.txt \\ \hline
tac & Vacia e imprime archivos en reversa. & Ej. tac notas.txt \\ \hline \hline
\end{tabular}
\section{Comandos para la gestión del sistema}
\begin {tabular}{||l|l||}
\hline \hline
Comando & Descripción  \\ \hline
exit & Cerrar la sesión actual.  \\ \hline
history & Muestra los comandos que han sido escritos por el usuario.  \\ \hline
logout & Salir del sistema.  \\ \hline
date & Muestra fecha y hora actuales. \\ \hline
clear & Elimina las líneas escritas en la terminal.\\ \hline 
echo & Repite líneas. \\ \hline \hline
\end{tabular}
\section{Comandos para apagado y reinicio del sistema}
\begin{tabular}{||l|l||} \hline \hline
Comando & Descripción \\ \hline
reboot & Reinicia la máquina. \\ \hline
shutdown & Apaga el sistema. \\ \hline 
init 0 & Apaga la máquina. \\ \hline \hline
\end{tabular}


% Nunca debe faltar esta última linea.
\end{document}
