% Ejemplo de documento LaTeX
% Tipo de documento y tamaño de letra
\documentclass[10pt]{article}
% Preparando para documento en Español.
% Para documento en Inglés no hay que hacer esto.
\usepackage[spanish]{babel}
\selectlanguage{spanish}
\usepackage[utf8]{inputenc}
% EL titulo, autor y fecha del documento
\title{Tabla de comandos Bash}
\author{Rios Quijada Danira}
\date{03 de Febrero de 2015}
% Aqui comienza el cuerpo del documento
\begin{document}
% Construye el título
\maketitle
\begin{tabular}{||l|l|l||}
\hline \hline
Comando & Descripción & Ejemplo \\
\hline
ls & ista del contenido del directorio. & Ej. ls-a \\ \hline
man & interfaz de los manuales de referencia electronicos. & Ej. man ls \\ \hline
echo & repite una linea de texto. & Ej. echo -n \\ \hline
pwd & muestra el nombre del directorio actual. & Ej. pwd -P \\ \hline
cd & cambiar de locacion. & Ej. cd notas \\ \hline
file & determina el tipo de archivo. & Ej. file-m \\ \hline
mkdir &crear un nuevo directorio. & Ej. mkdir notas \\ \hline
rmdir & eliminar dirctorios completos. & Ej. rmdir notas \\ \hline
touch & crear archivo en blanco. & Ej. touch notas.txt \\ \hline
cp & copia archivos y directorios. & Ej. notas.txt notas aaa \\ \hline
mv & mover o renombrar archivos. & Ej. mv notas.txt aaa \\ \hline
rm & remover directorios o archivos. & Ej. rm notas.txt \\ \hline
vi & editor de texto. &  \\ \hline
ZZ & salvar y salir. & \\ \hline
:w & salvar pero no salir.& \\ \hline
cat & vacia el contenido de un archivo. & Ej. cat notas.txt \\ \hline
less & vacia el contenido de archivos grandes. & \\ \hline
ls-l & ver los permisos de un archivo. & \\ \hline
chmod & muestra los permisos de un archivo y permite cambiarlos. & \\ \hline
head & muestra la primera parte de los archivos. & \\ \hline
tail & muestra la ultima parte de los archivos. & \\ \hline
sort & ordena las lineas de un archivo de texto. & \\ \hline
nl & muestra el numero de lineas de un archivo. & \\ \hline
cut & remueve secciones de cada linea del archivo. & \\ \hline
sed & editor para filtrar y transformar texto. & \\ \hline
uniq & reporta u omite lineas repetidas. & \\ \hline
tac & vacia e imprime archivos en reversa. & \\ \hline
history & muestra los ultimos comandos utilizados. & \\ \hline
reboot & reinicia el equipo. & \\ \hline
halt & apaga el equipo. & \\ \hline
chmod & se utiliza para cambiar los permisos de un archivo. & \\ \hline
clear & limpia la pantalla y coloca el prompt al principio de la misma. & \\ \hline \hline
\end{tabular}

% Nunca debe faltar esta última linea.
\end{document}
